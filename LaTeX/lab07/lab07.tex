\documentclass[12pt]{article}

\usepackage[utf8x]{inputenc}
\usepackage[T1, T2A]{fontenc}
\usepackage{fullpage}
\usepackage{multicol,multirow}
\usepackage{tabularx}
\usepackage{ulem}
\usepackage{listings} 
\usepackage[english,russian]{babel}
\usepackage{tikz}
\usepackage{pgfplots}
\usepackage{indentfirst}
\usepackage{ulem} 

\lstset{
	language=C++,
	basicstyle=\footnotesize\sffamily,
	showspaces=false,
	showstringspaces=false,
	showtabs=false,
	tabsize=4,
	breaklines=true,
	breakatwhitespace=false % переносить строки только если есть пробел
}
\parindent=1cm
\makeatletter
% \newcommand{\rindex}[2][\imki@jobname]{%
    % \index[#1]{\detokenize{#2}}%
% }
\newcommand{\print}[1]{{\large  \bf  #1} {\scriptsize \lstinputlisting[language=C++]{../../lab07/#1}}}
\makeatother
\newcolumntype{P}[1]{>{\raggedbottom\arraybackslash}p{#1}}

\linespread{1}
\pgfplotsset{compat=1.16}
\begin{document}

\section*{\centering Лабораторная работа №\,6 по курсу:\\ Дискретный анализ}

Выполнил студент группы М8О-208Б-17 МАИ \,\, \textit{Милько Павел}.

\subsection*{Задача}

При помощи метода динамического программирования разработать алгоритм решения задачи, определяемой своим вариантом; оценить время выполнения
алгоритма и объём затрачиваемой оперативной памяти. Перед выполнением задания
необходимо обосновать применимость метода динамического программирования.
Разработать программу на языке C или C++, реализующую построенный алгоритм.
Формат входных и выходных данных описан в варианте задания.

\paragraph*{Вариант 3:} Количество чисел.\\

Задано целое число n. Необходимо найти количество натуральных (без нуля) чисел, которые меньше n по
значению {\bf и} меньше n лексикографически (если сравнивать два числа как строки), а так же делятся на m без остатка.


\subsection*{Информация}

Динамическое программирование -- это способ решения сложных задач, путём разбиения их на подзадачи.Он применим к задачам с оптимальной подструктурой, выглядящим как набор перекрывающихся подзадач, сложность которых чуть меньше исходной. В этом случае время вычислений, по сравнению с ``наивными'' методами, можно значительно сократить.

Основа способа состоит в том, чтобы не решать меньше подзадачи несколько раз, а затем объединить решения.\\

\noindent Этапы построения алгоритма решения подзадач:
\begin{itemize}
	\item  Описать структуру оптимального решения.
	\item Составить рекурсивное решение для нахождения оптимального решения.
\item 	Вычисление значения, соответствующего оптимальному решению, методом восходящего анализа.
\item 	Непосредственное нахождение оптимального решения из полученной на предыдущих этапах информации.
	
\end{itemize}

\subsection*{Метод решения}

Для решения моей задачи можно применить динамическое программирование, выделив следующие подзадачи:

Для задачи $n, m$, где $len_n$ -- количество разрядов числа $n$ подзадача будет состоять из количества чисел, удовлетворяющих условию  и по длине равных $len_n$ : $$T(n,m) = n~div~m - 10^{len_n-1}~div~m$$ Важно не забыть про границы отрезка, если число  $10^{len_n-1}$ делится на $m$ нацело, то его нужно добавить к результату. Следующая подзадача будет решаться для чисел $n~div~10$ и $m$.

\subsection*{Исходный код}


\print{main.cpp}

\subsection*{Генератор тестов}

\lstinputlisting[language=Python]{../../lab07/gen.py}
\subsection*{Выводы}

Динамическое программирование довольно распространено, в различных олимпиадах около половины задач решаются с помощью ДП. Всюду, где встречаются подзадачи, и где их можно легко выделить, динамическое программирование позволяет существенно ускорить работу программы. Метод является достаточно гибким, так как это не алгоритм, а метод построения алгоритмов.


















































\end{document}

