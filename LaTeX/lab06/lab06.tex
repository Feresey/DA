\documentclass[12pt]{article}

\usepackage[utf8x]{inputenc}
\usepackage[T1, T2A]{fontenc}
\usepackage{fullpage}
\usepackage{multicol,multirow}
\usepackage{tabularx}
\usepackage{ulem}
\usepackage{listings} 
\usepackage[english,russian]{babel}
\usepackage{tikz}
\usepackage{pgfplots}
\usepackage{indentfirst}
\usepackage{ulem} 

\lstset{language=C++,
	basicstyle=\footnotesize\sffamily
	numbersep=5pt,
	showspaces=false,
	showstringspaces=false,
	showtabs=false,
	tabsize=4,
	breaklines=true,
	breakatwhitespace=false % переносить строки только если есть пробел
}
\parindent=1cm
\makeatletter
% \newcommand{\rindex}[2][\imki@jobname]{%
    % \index[#1]{\detokenize{#2}}%
% }
\newcommand{\print}[1]{{\large  \bf  #1} {\scriptsize \lstinputlisting[language=C++]{../../lab06/#1}}}
\makeatother
\newcolumntype{P}[1]{>{\raggedbottom\arraybackslash}p{#1}}

\linespread{1}
\pgfplotsset{compat=1.16}
\begin{document}

\section*{\centering Лабораторная работа №\,6 по курсу:\\ Дискретный анализ}

Выполнил студент группы М8О-208Б-17 МАИ \,\, \textit{Милько Павел}.

\subsection*{Задача}

\noindent Необходимо разработать программную библиотеку на языке С или
С++, реализующую простейшие арифметические действия и проверку условий над
целыми неотрицательными числами. На основании этой библиотеки нужно составить
программу, выполняющую вычисления над парами десятичных чисел и выводящую
результат на стандартный файл вывода.\\


Список арифметических операций:
\begin{itemize}
	\item Сложение (+)
	\item Вычитание (-)
	\item Умножение (*)
	\item Возведение в степень (ˆ)
	\item Деление (/)
\end{itemize}



В случае возникновения переполнения в результате вычислений, попытки вычесть из мень\-шего числа большее, деления на ноль или возведении нуля в нулевую степень, про\-грамма должна вывести на экран строку Error.\\

Список условий:
\begin{itemize}
\item Больше (>)
\item Меньше (<)
\item Равно (=)
\end{itemize}

\newpage

\subsection*{Метод решения}

Необходимо написать реализацию простейших арифметических и логических  операций с длинными числами. В качестве внутреннего представления числа логично выбрать вектор, в который будут добавляться ``разряды'' длинного числа. Числа в векторе располагаются от младшего разряда к старшему, максимальное значение числа в одном разряде ограничено выбранным основанием системы счисления, я выбрал $10^4$.

\paragraph{Сложение}{\it Длина результата максимум на 1 больше наибольшей из длин чисел.}


Реализуется тривиально: начиная с младших разрядов числа (начало вектора), суммируем оба числа. Если результат больше чем основание системы счисления, то записывается остаток от деления результата на основание системы счисления. При этом целая часть прибавляется к старшему разряду.

\paragraph{Вычитание}{\it Длина результата не больше длины вычитаемого числа.}\\
Реализуется аналогично сложению: из большего вычитается меньшее, если при вычитании разрядов получается отрицательное число, то к нему прибавляется основание системы счисления и занимается единица из старшего разряда.

\paragraph{Умножение} {\it Длина результата не больше суммы длин множителей.}\\
Алгоритм вычисления такой же, как и для обычных чисел (по разрядам, столбиком), за исключением того случая, когда результат становится больше основания системы счисления. Тогда целую часть от деления результата надо прибавить к следующему результату, а остаток от деления прибавить к разряду с номером, равным сумме позиций умножаемых разрядов двух чисел.

\paragraph{Деление}{\it  Длина результата не больше длины делимого числа.}\\
Осуществляется уголком: выбираем количество старших разрядов делимого числа так, чтобы получившийся срез по длине был равен делителю. Затем находим с помощью бинарного поиска (на отрезке от 0 до основания системы счисления) максимально возможный множитель,такой, что разница между срезом и умноженным делителем минимальна и положительна (0 тоже допустим, это означает что срез меньше делителя). Разницу запоминаем для дальнейшего деления, множитель записываем в старший разряд ответа. В начало остатка от деления записываем следующий старший разряд делителя и продолжаем алгоритм, пока не дойдём до младшего разряда делителя.

\paragraph{Возведение в степень} {\it Длина результата изначально неизвестна.}\\
Использован алгоритм быстрого возведения в степень, который намного производительнее простого умножения.

\paragraph{Операции сравнения} Реализованы поразрядным сравнением элементов и длин.

\subsection*{Исходный код}


\print{BigInt.hpp}
\print{BigInt.cpp}
\print{main.cpp} 
%\newpage

\subsection*{Генератор тестов}
\lstinputlisting[language=python]{../../lab06/gen.py}

\subsection*{Выводы}

Длинная арифметика встречается довольно часто (в язык питон она даже встроена изначально) и полезно понимать, как происходят базовые операции с числами, которые не помещаются в типы с фиксированной длиной.

Моя реализация длинной арифметики далека от идеала, например в качестве основания системы счисления можно было выбрать степень двойки. Это позволило бы заменить операции умножения и деления (при оперировании отдельными разрядами) на битовую конъюнкцию и битовый сдвиг соответственно. Так же в моей реализации не рассматриваются знаковые числа, а уж про дробную часть и говорить нечего. 

В целом лабораторная работа была интересной и несложной. Но что-то мне подсказывает, что дальше будет намного сложнее и больнее. 

\end{document}

