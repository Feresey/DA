\documentclass[12pt]{article}

\usepackage[utf8x]{inputenc}
\usepackage[T1, T2A]{fontenc}
\usepackage{fullpage}
\usepackage{multicol,multirow}
\usepackage{tabularx}
\usepackage{ulem}
\usepackage{listings} 
\usepackage[english,russian]{babel}
\usepackage{tikz}
\usepackage{pgfplots}
\usepackage{indentfirst}
\usepackage{ulem} 


\parindent=1cm
\makeatletter
\makeatother

\linespread{1}
\pgfplotsset{compat=1.16}
\begin{document}

\section*{\centering Лабораторная работа №\,5 по курсу:\\ Дискретный анализ}

Выполнил студент группы М8О-208Б-17 МАИ \,\, \textit{Милько Павел}.

\subsection*{Условие}
\paragraph*{Вариант 3:} Поиск образца с использованием статистики совпадений.
Найти образец в тексте используя статистику совпадений.

\textbf{\textit{Входные данные:}} на первой строке располагается образец, на второй ---
текст.

\textbf{\textit{Выходные данные:}} последовательность строк содержащих в себе номера
позиций, начиная с которых встретился образец. Строки должны быть
отсортированы в порядке возрастания номеров.

\subsection*{Метод решения}



\subsection*{Выводы}

\end{document}

